% % % % % % % % % % % % % % % % % % % % % % % % % % % % % % % % % % % % %

%\section*{Motivation} 
\newcommand{\ltx}[1]{{\tt \textbackslash#1}}
\begin{frame}{Background and Current Position}
\uncover<1->{
\begin{block}{Background}
\begin{itemize}
\item Born in Khulna, Bangladesh
\item B.Sc. in Mechanical Engineering from Bangladesh University of Engineering \&Technology
%\item Worked for Metso Inc. as a Design Engineer
\item M.Sc. in Computational Mechanics from University of Duisburg-Essen,Germany
\end{itemize}
\end{block}
}
\uncover<1->{
\begin{block}{Current Position}
\begin{itemize}
\item Joined WUT as ESR 14 on October, 2013
\item Supervised by Prof. Jacek Szumbarski
\end{itemize}
\end{block}
}
\end{frame}

 %%%%%%%%%%% Objectives %%%%%%%%%%%
\begin{frame}{Objectives}
\begin{block}{Primary Goal}
\begin{itemize}
\item Adjoint solver for unsteady Navier-Stokes (including check pointing)
%\item Usage of adjoint calculation for shape optimization and optionally for optimal control
\item Shape optimization using unsteady adjoint solver with option for optimal control
\item Optimize the performance with grid adaptation and optionally multi-gird
%\item Performance optimization with grid adaptation optional objective with multi-grid
\end{itemize}
\end{block}
\end{frame}

\begin{frame}{Work Plan}
\uncover<1->{
\begin{block}{Preliminaries}
\begin{itemize}\item Unsteady adjoint for a simple ODE with check pointing
\item optimal control for simple ODE
\end{itemize}
\end{block}
}
\uncover<1->{
\begin{block}{Prerequisites}
\begin{itemize}
\item Unsteady Euler/NS solver (Residual Distribution Scheme)
\end{itemize}
\end{block}
}
\uncover<1->{
\begin{block}{Primary Task}
\begin{itemize}
\item Implementation of unsteady adjoint in the NS solver
\item Implementation of moving geometry
\item Gradient-based shape optimization
\end{itemize}
\end{block}
}
\end{frame}

\begin{frame}{Work Plan - continued}
\begin{block}{Testing and evaluation}
\begin{itemize}
\item Testing of the Euler/NS code on non-stationary cases
\item Testing of the obtained gradients against finite difference
\item Testing of a complete optimization
\end{itemize}
\end{block}
\begin{block}{Performance improvements}
\begin{itemize}
\item Hessian-of-solution based mesh refinement
\item Goal-oriented based mesh refinement
\end{itemize}
\end{block}
\end{frame}


\begin{frame}
\frametitle{Progress}
	\begin{block}{Training}
		\begin{itemize}
			\item Literature review
			\item In house training on adjoint-based mesh adaptation and optimization
			\item Training workshop attended on AD tools organized by AboutFlow project
			\item Training workshop attended on MPI organized by HLRS, Germany
		\end{itemize}
	\end{block}
	\begin{block}{Current Status}
		\begin{itemize}
			\item Development of Adjoint implementation on Euler solver is underway.
		\end{itemize}	
	\end{block}
\end{frame}

\begin{frame}{Secondments}

\begin{block}{RWTH}
\begin{itemize}
\item Envisaged Feb-March 2015
\item Familiarization in parallelization of adjoint solvers
\item Review of AD tools for parallel application
\item Implementation of operator overloading based AD to develop Adjoint solver 
\item Participation in courses on AD and scientific computing
\end{itemize}
\end{block}
\end{frame}
\begin{frame}{Secondment}
\begin{block}{Rolls Royce}
\begin{itemize}
\item Planned October 2016
\item Familiarization with the industrial flow cases
\item Training on industrial approach to optimization
\item Investigation of turbo-machinery problem using the developed adjoint solver 
\item Application of grid adaptation tool chain available in WUT for turbo-machinery test cases
\item Benchmarking of developed approaches with current practices
\end{itemize}
\end{block}
\end{frame}

\begin{frame}{Expected Outcome of the Project}
\begin{block}{}
\begin{itemize}
\item Robust adjoint solver with shape optimization and grid adaptation capability
\item Experience gain in cfd solver development and application to industrial flow problems
\item Collaboration with academia and industry

\end{itemize}
\end{block}
\end{frame}

%%%%%%%%%%% Formulation of Equation %%%%%%%%%%%
%\begin{frame}{Duality Formulation For Adjoint Design}
%\begin{block}{Primal:}
%\begin{equation}
%Q^{n+1}=F(Q^n)
%\end{equation}
%\end{block}
%\begin{block}{Tangent Linear:}
%\begin{equation}
%\frac{\partial Q^{n+1}}{\partial \alpha} = \frac{\partial F}{\partial Q}\frac{\partial Q}{\partial \alpha} +\frac{\partial I}{\partial \alpha}
%\end{equation}
%\end{block}
%
%\begin{block}{Adjoint Equation:}
%\begin{equation}
%v^n = \highlight{\left(\frac{\partial F}{\partial Q}\right)^{T}}v^{n+1} +\frac{\partial I}{\partial Q}
%\end{equation}
%\end{block}
%\end{frame}

\begin{frame}{Advantage of Finite Difference to Develop Jacobian}
\tiny
%\begin{equation}
%\begin{bmatrix}
%a_{11}  & a_{12}  & 0   & \cdots & 0 \\
%a_{21}  & a_{22}  & a_{23}  & \ddots  & \vdots \\
%0 & a_{32}  & a_{33}   & \ddots  & \vdots \\
%%\vdots & \ddots & \ddots & \ddots & \ddots & \ddots &  & \vdots \\
%\vdots &  & \ddots  & \ddots& \vdots\\
%%\vdots  &  & & \ddots & a_{76}  & a_{77}  &  a_{78}  & 0\\
%%\vdots  &  & & & \ddots & a_{87}  & a_{88}  &  a_{89}\\
%0   & \cdots & 0 & a_{98} & a_{99}  \\
%\end{bmatrix}
%\end{equation}

%\begin{equation}
%\begin{bmatrix}
%a_{11}  & a_{12}  & 0 & \cdots & \cdots & \cdots & \cdots & 0 \\
%a_{21}  & a_{22}  & a_{23}  & \ddots & && & \vdots \\
%0 & a_{32}  & a_{33} & a_{34}  & \ddots & &  & \vdots \\
%\vdots & \ddots & \ddots & \ddots & \ddots & \ddots &  & \vdots \\
%\vdots & & \ddots & \ddots & \ddots & \ddots & \ddots& \vdots\\
%\vdots  &  & & \ddots & a_{76}  & a_{77}  &  a_{78}  & 0\\
%\vdots  &  & & & \ddots & a_{87}  & a_{88}  &  a_{89}\\
%0 & \cdots &  \cdots & \cdots & \cdots & 0 & a_{98} & a_{99}  \\
%\end{bmatrix}
%\end{equation}

\begin{equation}
\begin{bmatrix}
a_{11}  & a_{12}  & 0  & \cdots & \cdots & \cdots & 0 \\
a_{21}  & a_{22}  & a_{23}  & \ddots & &&  \vdots \\
0 & a_{32}  & a_{33} & a_{34}  & \ddots &   & \vdots \\
\vdots & & \ddots & \ddots & \ddots & \ddots& \vdots\\
\vdots  &   &  & a_{76}  & a_{77}  &  a_{78}  & 0\\
\vdots  &   & & \ddots & a_{87}  & a_{88}  &  a_{89}\\
0 & \cdots &  \cdots & \cdots & 0 & a_{98} & a_{99}  \\
\end{bmatrix}
\end{equation}
\normalsize
\begin{itemize}
\item Developed Jacobian is a complete sparse matrix which can be calculated locally
\item It is conveniently scalable
\item Options to implement higher order FDM, if more accurate jacobian is required
\item Tested for stationary problems with sufficient accuracy
\end{itemize}
\end{frame}


%\begin{frame}{Introduction, Table of Contents, Roadmap}
%  \uncover<+->{beamer provides a nice Table of Content
%    feature if you use the \ltx{section} command.}
%
%  \uncover<+->{There is little point in starting
%   your talk with a boring ToC, the audience will not know
%   what these headings mean. Much better to first give an 
%   overview/motivation, then give them a 'map' how you will be explaining
%   your work.}
%
%  \uncover<+->{All sections will appear in section listings on the
%    borders of the style (here at the very top). Since these listings
%    are to help orientate the audience, I don't think this is useful
%    before you've shown the ToC, hence it is commented out in the
%    source of this introductory part.}
%
%  \uncover<+->{Note, you could suppress the appearance of a section title
%   in the ToC if you use the \ltx{section*} command,
%   but it would still appear at the top of each frame.}
%\end{frame}

%\begin{frame}{What application are we aiming for?}
%  Example of itemisation coming up as a block using
%  the {\tt \textbackslash uncover} command:
%  \uncover<2->{
%  \begin{itemize}
%  \item All items
%  \item are shown
%  \item at once,
%  \end{itemize}
%  }
%
%  \uncover<3->{
%    Or items popping up one after the other:
%    \begin{itemize}
%    \item<4-> Note that when you look at the source of this frame, 
%    \item<6-> we explicitly number how things
%    \item<5-> come up.
%    \end{itemize}
%
%    \slidefoot{It is not very useful to give detailed lists of references
%      at the end, the audience can't flip back and forth in your
%      presentation. Better to provide a shortened reference at the relevant 
%      page:\\
%      T.~Tantau et al., ``The BEAMER class'', Dec 2013, tug.ctan.org }
%  }
%\end{frame}

%\begin{frame}{What is the problem that we address?}
%  \uncover<+->{While here}
%  \begin{itemize}
%  \item<+->we simply
%  \item <+-> increment.
%  \end{itemize}
%
%  \uncover<+->{But beware:}
%  \begin{itemize}
%  \item<+-> Avoid
%  \item<+-> uncovering 
%  \item<+-> too little
%  \item<+-> information
%  \item<+-> in too many
%  \item<+-> small 
%  \item<+-> steps
%  \end{itemize}
%\end{frame}

%\begin{frame}{Ready for the roadmap?}
%  Now that we set our work in context, we can show the Table 
%  of Contents.
%\end{frame}
